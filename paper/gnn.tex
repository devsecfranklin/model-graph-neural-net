% !TeX TXS-program:compile = txs:///pdflatex/[--shell-escape]

\documentclass{article}
\usepackage{blindtext}

%%% FONTS %%%
\usepackage[T1]{fontenc}
\usepackage[utf8]{inputenc}
\usepackage[english]{babel}
%\usepackage{tgpagella} % set the document font to TeX Gyre Pagella
%\usepackage{tgbonum} % set the document font to TeX Gyre Bonum
%\usepackage{fontawesome5} % The Creative Commons icons

%%% DRAFT watermark %%%
\usepackage{draftwatermark}
\SetWatermarkText{DRAFT}
\SetWatermarkScale{5.1}
\SetWatermarkLightness{0.8}

\usepackage{xcolor} % \textcolor{red}{text} will be red for notes
\definecolor{lightgray}{gray}{0.6}
\definecolor{medgray}{gray}{0.4}

\usepackage{hyperref}
\hypersetup{
	colorlinks=true,
	urlcolor= blue,
	citecolor=blue,
	linkcolor= blue,
	%bookmarks=true,
	%bookmarksopen=false,
}

% Code to add paragraph numbers and titles
\newif\ifptitle
\newif\ifpnumber
\newcounter{para}
\newcommand\ptitle[1]{\par\refstepcounter{para}
	{\ifpnumber{\noindent\textcolor{lightgray}{\textbf{\thepara}}\indent}\fi}
	{\ifptitle{\textbf{[{#1}]}}\fi}}
\ptitletrue  % comment this line to hide paragraph titles
\pnumbertrue  % comment this line to hide paragraph numbers

% for inserting images in line
\usepackage{graphicx}
\graphicspath{ {code/} }

%\usepackage[verbose]{wrapfig} % wrap text around image

\definecolor{myblue}{HTML}{4285F4} % color table cells

\usepackage[tablegrid]{vhistory} % version history package


% Allows to rewrite the same title in the supplement
\newcommand{\mytitle}{Analysis and Optimization of Security Infrastructure with Deep Learning Methods}
\usepackage{authblk}
\usepackage{comment} % allows block comments

\usepackage{ragged2e} % use flush and justify for text blocks
\usepackage{csquotes} % use \displayquote{} in the doc

%%% GRAPHICS  AND CODE BLOCKS %%%
\usepackage[listings, minted]{tcolorbox}
\usepackage{xcolor,colortbl}
\definecolor{myblue}{RGB}{0,163,243}
\definecolor{mygrey}{RGB}{128,128,128}
\definecolor{whitesmoke}{RGB}{245,245,245}
\newtcolorbox[auto counter, number within=section]{mybox}[2][]{
	colbacktitle=mygrey,
	colback=whitesmoke,
	title={#2},
	fonttitle=\ttfamily\small,
	fontupper=\sffamily\small,
	halign=flush left,
	rounded corners
}

% Headers and footers
\usepackage{fancyhdr}
\pagestyle{fancy}
\fancyhf{}
\lhead{\mytitle}
\lfoot{\tiny{November 20, 2021}}
\rfoot{\tiny{version: \vhCurrentVersion}}

\usepackage{glossaries}
	\makeglossaries

\newacronym{Edge}{Edge}{define edge}
\newacronym{GitOps}{GitOps}{Implementing  version control, collaboration, compliance, and CI/CD tooling for infrastructure automation}
\newacronym{MPNN}{Message Passing Nerual Network}{define it}
\newacronym{Node}{Node}{define a node}
\newacronym{Scalar}{Scalar}{define a scalar here}
\newacronym{Vector}{Vector}{define a vector here}

\newcommand*{\myglossaryindent}{0.65cm}
\newcommand*{\myglsdescwidth}{10cm}



\begin{document}
\vspace{2mm}
\title{\mytitle}
\author[1,2]{Franklin E. Diaz\\ \texttt\href{emailto: fdiaz@paloaltonetworks.com}{fdiaz@paloaltonetworks.com}}
\affil[1]{Palo Alto Networks}
\affil[2]{Professional Services - Automation}
\begin{titlepage}
	\maketitle
\begin{abstract}
	This paper describes a project.
\end{abstract}
\end{titlepage}

\begin{comment}
Source files for this paper are available at: \url{https://github.com/devsecfranklin/model-graph-neural-net/tree/main/paper}
\end{comment}

\vspace{2mm}
\section{\label{sec:Introduction}\mytitle}

\justifying
The Terraform software made by Hashicorp is typically used as a means of automating network infrastructure
deployments. Our teams produce Terraform modules and code frequently.

\section{\label{sec:Graph}About Graph Neural Networks}

\justifying
While an in depth explanation of graph theory is beyond the scope of this paper, some background
will be provided here to make the paper accessible to a wider audience. A collection of related references
will be included at the end of this paper for those who may be interested.

\section{\label{sec:collection}Gathering Data}

\justifying
Once a Terraform code base has been initialized, the user has the option to generate a directed graph of the infrastructure. Consider the following declaration of a Cloud Function for Google Cloud.

\begin{mybox}{\thetcbcounter: Terraform Declaration Example}
    \lstinputlisting{code/main.tf}
\end{mybox}


\justifying
\href{https://graphviz.org/}{Graphviz is open source graph visualization software}. Graph visualization is a way of representing
structural information as diagrams of abstract graphs and networks. DOT is a language used to describe graphs. Conversion of
Terraform HCL to a directed graph representation is trivial thanks to Graphviz and Dot.

\justifying
\begin{figure}[H]
	\includegraphics[width=12cm]{graph.png}
	\caption{DOT output from Terraform Files as a digraph}
	\label{pygraph}
\end{figure}

\justifying
Once our Terraform files have been output as a directed graph, we need to bring it in to Python so we can
perform operations on it. To run Terraform commands in an automated fashion, \href{https://pypi.org/project/python-terraform/}{the python-terraform module}
is used.

\justifying
Directed graphs are ingested into Python as an object using the networkx module. Now we are ready to
perform operations on the digraph, displayed in figure \ref{digraph}.

\justifying
\begin{figure}[H]
    \includegraphics[width=12cm]{graph3.png}
    \caption{Directed graph as ingested by Networx}
    \label{digraph}
\end{figure}

Note that new numerical labels are displayed in place of the original labels. These original labels are
stored in the node objects in case we need them later. The root node of the adjacency matrix
is labeled as the 0th element in the graph.

\subsection{\label{sec:adjacency}The Adjacency Matrix}

This module also provides
an easy means of converting our graph into an adjacency matrix, which is a mathematical representation of
our graph. The nodes are represented in a columnar format known as a matrix.


features


\clearpage
\begin{versionhistory}
	\vhEntry{v0.1}{Nov. 20th, 2021}{Franklin Diaz}{Initial Draft}
\end{versionhistory}
\nocite{*}
\bibliographystyle{plain}
\bibliography{mybib.bib}

\end{document}
