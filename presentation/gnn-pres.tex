% !TeX TXS-program:compile = txs:///pdflatex/[--shell-escape]
% !TeX root = gnn-pres
% !TeX encoding = UTF-8
% !TeX spellcheck = en_US
% https://orcid.org/0000-0003-4586-8500


%\documentclass[14pt]{beamer}
\documentclass[aspectratio=169]{beamer} % Other possible values are: 1610, 149, 54, 43 and 32. By default, it is to 128mm by 96mm(4:3)

\documentclass{article}
\usepackage{blindtext}

%%% FONTS %%%
\usepackage[T1]{fontenc}
\usepackage[utf8]{inputenc}
\usepackage[english]{babel}
%\usepackage{tgpagella} % set the document font to TeX Gyre Pagella
%\usepackage{tgbonum} % set the document font to TeX Gyre Bonum
%\usepackage{fontawesome5} % The Creative Commons icons

%%% DRAFT watermark %%%
\usepackage{draftwatermark}
\SetWatermarkText{DRAFT}
\SetWatermarkScale{5.1}
\SetWatermarkLightness{0.8}

\usepackage{xcolor} % \textcolor{red}{text} will be red for notes
\definecolor{lightgray}{gray}{0.6}
\definecolor{medgray}{gray}{0.4}

\usepackage{hyperref}
\hypersetup{
	colorlinks=true,
	urlcolor= blue,
	citecolor=blue,
	linkcolor= blue,
	%bookmarks=true,
	%bookmarksopen=false,
}

% Code to add paragraph numbers and titles
\newif\ifptitle
\newif\ifpnumber
\newcounter{para}
\newcommand\ptitle[1]{\par\refstepcounter{para}
	{\ifpnumber{\noindent\textcolor{lightgray}{\textbf{\thepara}}\indent}\fi}
	{\ifptitle{\textbf{[{#1}]}}\fi}}
\ptitletrue  % comment this line to hide paragraph titles
\pnumbertrue  % comment this line to hide paragraph numbers

% for inserting images in line
\usepackage{graphicx}
\graphicspath{ {code/} }

%\usepackage[verbose]{wrapfig} % wrap text around image

\definecolor{myblue}{HTML}{4285F4} % color table cells

\usepackage[tablegrid]{vhistory} % version history package


% Allows to rewrite the same title in the supplement
\newcommand{\mytitle}{Analysis and Optimization of Security Infrastructure with Deep Learning Methods}
\usepackage{authblk}
\usepackage{comment} % allows block comments

\usepackage{ragged2e} % use flush and justify for text blocks
\usepackage{csquotes} % use \displayquote{} in the doc

%%% GRAPHICS  AND CODE BLOCKS %%%
\usepackage[listings, minted]{tcolorbox}
\usepackage{xcolor,colortbl}
\definecolor{myblue}{RGB}{0,163,243}
\definecolor{mygrey}{RGB}{128,128,128}
\definecolor{whitesmoke}{RGB}{245,245,245}
\newtcolorbox[auto counter, number within=section]{mybox}[2][]{
	colbacktitle=mygrey,
	colback=whitesmoke,
	title={#2},
	fonttitle=\ttfamily\small,
	fontupper=\sffamily\small,
	halign=flush left,
	rounded corners
}

% Headers and footers
\usepackage{fancyhdr}
\pagestyle{fancy}
\fancyhf{}
\lhead{\mytitle}
\lfoot{\tiny{November 20, 2021}}
\rfoot{\tiny{version: \vhCurrentVersion}}

\usepackage{glossaries}
	\makeglossaries

\newacronym{Edge}{Edge}{define edge}
\newacronym{GitOps}{GitOps}{Implementing  version control, collaboration, compliance, and CI/CD tooling for infrastructure automation}
\newacronym{MPNN}{Message Passing Nerual Network}{define it}
\newacronym{Node}{Node}{define a node}
\newacronym{Scalar}{Scalar}{define a scalar here}
\newacronym{Vector}{Vector}{define a vector here}

\newcommand*{\myglossaryindent}{0.65cm}
\newcommand*{\myglsdescwidth}{10cm}



% Table of Contents at Section start
\AtBeginSection[]
{
    \begin{frame}
        \frametitle{\bfseries\Huge\textcolor{black}{.}}
        \tableofcontents[currentsection]
    \end{frame}
}

\begin{document}

%%% Title Slide %%%
\usebackgroundtemplate{\includegraphics[width=\paperwidth]{../images/landscape.jpg}}
\begin{frame}
    \setlength{\TPHorizModule}{\textwidth}
    \setlength{\TPVertModule}{\textwidth}
    \begin{textblock}{0.6} (0.05,0.05)
      \bfseries\Huge\textcolor{cyan}{Learning About Deep Learning, and Maybe a Few Other Things}
    \end{textblock}
    % \begin{textblock}{WIDTH}(XCOORDINATE,YCOORDINATE)
    \begin{textblock}{0.5} (0.65,0.52)
        \bfseries\textcolor{green}{Franklin Diaz, Cosmic Voyager}
    \end{textblock}
\end{frame}

\note{My presentation title is here. This talk is about my on-going journey into the world of graph theory and neural networks. This is a description of the learning process as well as the project and presenting some results.}

% Intro Section
\usebackgroundtemplate{\includegraphics[width=\paperwidth]{../images/field.jpg}}
\section{Introduction}
\note{This first section is an Introduction to me and my project}

\usebackgroundtemplate{\includegraphics[width=\paperwidth]{../images/tree.jpg}}
\begin{frame}{}
    \setlength{\TPHorizModule}{\textwidth}
    \setlength{\TPVertModule}{\textwidth}
    % Slide title in upper left
    \begin{textblock}{0.74} (0.05,0.05)
        \bfseries\large\textcolor{white}{About Me}
    \end{textblock}

    \begin{columns}
        \begin{column}{0.5\textwidth}
            \begin{itemize}
                \item I am a Security Consultant at Palo Alto Networks, cloud and automation for past 2+ years
                \item Did Data Eng/DevSecOps at Salesforce for 5 years.
            \item Been going to security conferences for a while.
            \end{itemize}
        \end{column}
        \begin{column}{0.45\textwidth}
            \begin{center}
            \includegraphics[width=1.0\linewidth, height=0.7\textheight]{../images/me.jpg}
            \end{center}
        \end{column}
    \end{columns}
\end{frame}

\note[itemize]{
    \item Here is a picture of me, but modified by a popular local artist.
    \item In my current role I get to work with the major cloud providers.
    \item In the past I was not a Data Scientist, but did some time on the Security Data Engineering team at Salesforce. This gave me a bit of a head start with data pipelines, directed acyclic graphs, and a few other things.
}

\usebackgroundtemplate{\includegraphics[width=\paperwidth]{../images/tree.jpg}}
\begin{frame}{}
    \setlength{\TPHorizModule}{\textwidth}
    \setlength{\TPVertModule}{\textwidth}
    \begin{textblock}{0.74} (0.05,0.05)
        \bfseries\large\textcolor{white}{The Project}
    \end{textblock}
    \begin{itemize}
        \item Realized that Terraform can output directed graphs.
        \item Had done a lot of work at Salesforce with directed graphs, data pipeline orchestration with AirFlow, etc. so I was somewhat familiar with the output I was seeing.
        \item The first question I had was, what can I do with these directed graphs?
        \item My hunch was I could ``do some processing and analysis'' of all this security infrastructure graph data and hoped that could lead to... predictions?
    \end{itemize}
\end{frame}
\note[itemize]{
    \item In case you are not familiar, Terraform is software that allows you to declare resources like network elements in public cloud providers.
    \item Had and still have this vague notion that if I had enough data I could find ``outliers''. Maybe like a modernized version of a Pareto analysis?
}

\usebackgroundtemplate{\includegraphics[width=\paperwidth]{../images/tree.jpg}}
\begin{frame}{}
    \setlength{\TPHorizModule}{\textwidth}
    \setlength{\TPVertModule}{\textwidth}
    % Slide title in upper left
    \begin{textblock}{0.74} (0.05,0.05)
        \bfseries\large\textcolor{white}{What's a DiGraph?}
    \end{textblock}

    \begin{columns}
        \begin{column}{0.5\textwidth}
            \begin{center}
                \includegraphics[width=1.0\linewidth,height=0.7\textheight]{../images/digraph-anatomy.png}
            \end{center}
         \end{column}
         \begin{column}{0.5\textwidth}
             \begin{center}
                \includegraphics[width=1.0\linewidth,height=0.7\textheight]{../images/strong-components.png}
             \end{center}
        \end{column}
    \end{columns}
\end{frame}

\note[itemize]{
    \item The big takeaway here is the idea of ``edges'' and ``nodes''
    \item \href{https://algs4.cs.princeton.edu/42digraph/}{Source: Algorithms, 4th Edition, by Robert Sedgewick and Kevin Wayne}
    \item \href{https://www.youtube.com/watch?v=mXoiHgH4mEE}{Wrath of Math!}
}

\usebackgroundtemplate{\includegraphics[width=\paperwidth]{../images/field.jpg}}
\section{What is Deep Learning, Exactly?}

\usebackgroundtemplate{\includegraphics[width=\paperwidth]{../images/tree.jpg}}
\begin{frame}{}
    \setlength{\TPHorizModule}{\textwidth}
    \setlength{\TPVertModule}{\textwidth}
    % Slide title in upper left
    \begin{textblock}{0.74} (0.05,0.05)
        \bfseries\large\textcolor{white}{The Rise of Deep Learning}
    \end{textblock}
    \bigskip
    \includegraphics[width=1.0\linewidth,height=0.7\textheight]{../images/dl_timeline.png}

\end{frame}

\note[itemize]{
    \item GPUs have made it possible to expand accessibility to DL
    \item the CUDA toolkit from Nvidia has made things easier for researchers.
}

\usebackgroundtemplate{\includegraphics[width=\paperwidth]{../images/tree.jpg}}
\begin{frame}{}
    \setlength{\TPHorizModule}{\textwidth}
    \setlength{\TPVertModule}{\textwidth}
    % Slide title in upper left
    \begin{textblock}{0.74} (0.05,0.05)
        \bfseries\large\textcolor{white}{Quick Intro to a Giant Topic}
    \end{textblock}
    \bigskip
    \includegraphics[width=1.0\linewidth,height=0.7\textheight]{../images/Diff-ML-DL.jpg}

\end{frame}

\note[itemize]{
    \item \href{https://www.researchgate.net/publication/338585724_Improved_Approach_for_Identification_of_Real_and_Fake_Smile_using_Chaos_Theory_and_Principal_Component_Analysis}{image source/credit}
    \item ML feature extraction can be a huge undertaking, up to 80\% of a project.
    \item DL attempts to automatically learn features that are most useful for a task from raw data.
    \item The nodes in a digraph are ``neurons'' or ``units'' in the DL/graph theory context.
    \item The neurons perform two steps. They calculate a ``weighted sum'' and pass the result through an ``activation function'' such as a rectifier activation function.
    \item These neurons or units that go through the rectifier function are called ``RelUs'' for short. Lot's of descriptive info in this one term!
    \item Depth of the GNN is measured by the number of connected layers.
    \item DL needs very large data sets for accurate feature determination. Data sets with lots of features are known as ``high density''.
    \item We humans interpret the features and output based on what we are trying to model.
}

\usebackgroundtemplate{\includegraphics[width=\paperwidth]{../images/tree.jpg}}
\begin{frame}{}
    \setlength{\TPHorizModule}{\textwidth}
    \setlength{\TPVertModule}{\textwidth}
    % Slide title in upper left
    \begin{textblock}{0.74} (0.05,0.05)
        \bfseries\large\textcolor{white}{Amazing Tools Available}
    \end{textblock}
\end{frame}

\note[itemize]{
    \item Google Deep Learning Container Images
    \item Continuous Machine Learning (CML) Project
    \item Kaggle and shared Jupyter Notebooks
}

\usebackgroundtemplate{\includegraphics[width=\paperwidth]{../images/field.jpg}}
\section{The Journey}

\usebackgroundtemplate{\includegraphics[width=\paperwidth]{../images/tree.jpg}}
\begin{frame}{}
    \setlength{\TPHorizModule}{\textwidth}
    \setlength{\TPVertModule}{\textwidth}
    % Slide title in upper left
    \begin{textblock}{0.74} (0.05,0.05)
        \bfseries\large\textcolor{white}{What Have I Gotten Myself Into?}
    \end{textblock}
    % main body bullet points
    \begin{itemize}
        \item This is an example of a list.
        \item Important business information.
    \end{itemize}
\end{frame}

\usebackgroundtemplate{\includegraphics[width=\paperwidth]{../images/tree.jpg}}
\begin{frame}{}
    \setlength{\TPHorizModule}{\textwidth}
    \setlength{\TPVertModule}{\textwidth}
    % Slide title in upper left
    \begin{textblock}{0.74} (0.05,0.05)
        \bfseries\large\textcolor{white}{Yak Shaving, Side Quests, Endless Rabbit Holes}
    \end{textblock}
    % main body
    \begin{itemize}
        \item Makefiles and GNU Autotools
        \item NVIDA Jetson Nano as cluster nodes
        \item SLURM cluster scheduler
        \item OpenMPI for parallel builds
        \item Docker and Containers
        \item k8s and Rancher k3s
        \item Data Version Control \href{https://dvc.org}{dvc.org}
        \item Storing/accessing data in GCP buckets
        \item Continuous Machine Learning \href{https://cml.dev/}{cml.dev}
        \item Internal Pypi and Debian/Raspbian mirror (used too much bandwidth on home connection)
    \end{itemize}
\end{frame}

\defverbatim[colored]\lstI{
\begin{lstlisting}[language=Python,basicstyle=\ttfamily,keywordstyle=\color{red}]

# Generate a PNG from Terraform
terraform graph | dot -Tpng > graph.png

# Generate vector graphic from Terraform
terraform graph | dot -Tsvg -o graph.svg
    \end{lstlisting}
}

\usebackgroundtemplate{\includegraphics[width=\paperwidth]{../images/tree.jpg}}
\begin{frame}{}
    \setlength{\TPHorizModule}{\textwidth}
    \setlength{\TPVertModule}{\textwidth}
    % Slide title in upper left
    \begin{textblock}{0.74} (0.05,0.05)
        \bfseries\large\textcolor{white}{Dot Data Collection}
    \end{textblock}
    % main body
    A big barrier to entry was removed by the ability to output
    a Directed Graph from Terraform. \href{https://youtu.be/2FytVBJHUKk}{Click for video}

    \lstI

\end{frame}

\usebackgroundtemplate{\includegraphics[width=\paperwidth]{../images/tree.jpg}}
\begin{frame}{}
    \setlength{\TPHorizModule}{\textwidth}
    \setlength{\TPVertModule}{\textwidth}
    % Slide title in upper left
    \begin{textblock}{0.74} (0.05,0.05)
        \bfseries\large\textcolor{white}{Python Data Collection}
    \end{textblock}
    This became the basis for collection via Python.

    \href{https://youtu.be/aTE8uZVO248}{Click for video} %  % 2.5x python collection
\end{frame}

\usebackgroundtemplate{\includegraphics[width=\paperwidth]{../images/tree.jpg}}
\begin{frame}{}
    \setlength{\TPHorizModule}{\textwidth}
    \setlength{\TPVertModule}{\textwidth}
    % Slide title in upper left
    \begin{textblock}{0.74}(0.05,0.05)
        \bfseries\large\textcolor{white}{Data Processing}
    \end{textblock}

    \begin{columns}
    \begin{column}{0.5\textwidth}
        \begin{center}
            \includegraphics[width=1.0\linewidth,height=0.7\textheight]{../images/kaggle-cml.png}
        \end{center}
    \end{column}
    \begin{column}{0.5\textwidth}
        \begin{center}
            \includegraphics[width=1.0\linewidth,height=0.7\textheight]{../images/strong-components.png}
        \end{center}
    \end{column}
\end{columns}

\end{frame}

\usebackgroundtemplate{\includegraphics[width=\paperwidth]{../images/tree.jpg}}
\begin{frame}{}
    \setlength{\TPHorizModule}{\textwidth}
    \setlength{\TPVertModule}{\textwidth}
    % Slide title in upper left
    \begin{textblock}{0.74} (0.05,0.05)
        \bfseries\large\textcolor{white}{Data Collection Container}
    \end{textblock}

\end{frame}

\usebackgroundtemplate{\includegraphics[width=\paperwidth]{../images/tree.jpg}}
\begin{frame}{}
    \setlength{\TPHorizModule}{\textwidth}
    \setlength{\TPVertModule}{\textwidth}
    % Slide title in upper left
    \begin{textblock}{0.74}(0.05,0.05)
        \bfseries\large\textcolor{white}{Data Storage - Google Cloud}
    \end{textblock}

Data storage with GCP because it's (relatively) easy.
\end{frame}

\usebackgroundtemplate{\includegraphics[width=\paperwidth]{../images/tree.jpg}}
\begin{frame}{}
    \setlength{\TPHorizModule}{\textwidth}
    \setlength{\TPVertModule}{\textwidth}
    % Slide title in upper left
    \begin{textblock}{0.74}(0.05,0.05)
        \bfseries\large\textcolor{white}{Data Storage - Kaggle}
    \end{textblock}

    Data storage and tagging using DVC

\end{frame}

\usebackgroundtemplate{\includegraphics[width=\paperwidth]{../images/tree.jpg}}
\begin{frame}{}
    \setlength{\TPHorizModule}{\textwidth}
    \setlength{\TPVertModule}{\textwidth}
    % Slide title in upper left
    \begin{textblock}{0.74}(0.05,0.05)
        \bfseries\large\textcolor{white}{Data Storage - DVC}
    \end{textblock}

Data storage and tagging using DVC

\end{frame}

\usebackgroundtemplate{\includegraphics[width=\paperwidth]{../images/tree.jpg}}
\begin{frame}{}
    \setlength{\TPHorizModule}{\textwidth}
    \setlength{\TPVertModule}{\textwidth}
    % Slide title in upper left
    \begin{textblock}{0.74}(0.05,0.05)
        \bfseries\large\textcolor{white}{Data Tagging}
    \end{textblock}

\end{frame}

\usebackgroundtemplate{\includegraphics[width=\paperwidth]{../images/tree.jpg}}
\begin{frame}{}
    \setlength{\TPHorizModule}{\textwidth}
    \setlength{\TPVertModule}{\textwidth}
    % Slide title in upper left
    \begin{textblock}{0.74}(0.05,0.05)
        \bfseries\large\textcolor{white}{Data Pipeline}
    \end{textblock}

    \begin{columns}
    \begin{column}{0.5\textwidth}
        \begin{itemize}
            \item The Data Pipeline is a set of processes that move and transform data from various sources to a destination where new value can be derived.
            \item THe DP is the foundation of analytics, reporting, and machine learning capabilities.
        \end{itemize}
    \end{column}
    \begin{column}{0.45\textwidth}
        \begin{center}
            \includegraphics[width=1.0\linewidth, height=0.7\textheight]{project.png}
        \end{center}
    \end{column}
\end{columns}
\end{frame}

\note[itemize]{
    \item Source: Data Pipelines pocket reference p1-2
}

\usebackgroundtemplate{\includegraphics[width=\paperwidth]{../images/field.jpg}}
\section{Visualizations}

\usebackgroundtemplate{\includegraphics[width=\paperwidth]{../images/tree.jpg}}
\begin{frame}{}
    \setlength{\TPHorizModule}{\textwidth}
    \setlength{\TPVertModule}{\textwidth}
    % Slide title in upper left
    \begin{textblock}{0.74}(0.05,0.05)
        \bfseries\large\textcolor{white}{Graphviz/Dot output}
    \end{textblock}

    \begin{textblock}{0.75}(0.025,0.08)
        \bigskip
        \includegraphics[width=1.0\linewidth,height=0.7\textheight]{../images/graph-dl-test}
    \end{textblock}
\end{frame}
\note[itemize]{
    \item This is the first thing I saw when I started converting the data.
    \item Was excited here since I was able to change the color of the nodes.
    \item Obviously this is not yet a usable result.
}

\usebackgroundtemplate{\includegraphics[width=\paperwidth]{../images/tree.jpg}}
\begin{frame}{}
    \setlength{\TPHorizModule}{\textwidth}
    \setlength{\TPVertModule}{\textwidth}
    % Slide title in upper left
    \begin{textblock}{0.74}(0.05,0.05)
        \bfseries\large\textcolor{white}{Gephi}
    \end{textblock}

\end{frame}

\usebackgroundtemplate{\includegraphics[width=\paperwidth]{../images/field.jpg}}
\section{So Now What?}

\usebackgroundtemplate{\includegraphics[width=\paperwidth]{../images/tree.jpg}}
\begin{frame}{}
    \setlength{\TPHorizModule}{\textwidth}
    \setlength{\TPVertModule}{\textwidth}
    % Slide title in upper left
    \begin{textblock}{0.74}(0.05,0.05)
        \bfseries\large\textcolor{white}{Useful Intermediate Results}
    \end{textblock}
    % main body bullet points
    \begin{itemize}
        \item Standardizing my data collection on JSON.
        \item Made some super cool functions for parsing nested JSON.
        \item Importing JSON to Pandas dataframes.
    \end{itemize}
\end{frame}
\note[itemize]{
    \item Tabular data in Pandas can be output in all kinds of formats.
    \item Pandas data frames can be the input for other Machine Learning tools and frameworks.
}

\usebackgroundtemplate{\includegraphics[width=\paperwidth]{../images/tree.jpg}}
\begin{frame}{}
    \setlength{\TPHorizModule}{\textwidth}
    \setlength{\TPVertModule}{\textwidth}
    % Slide title in upper left
    \begin{textblock}{0.74}(0.05,0.05)
        \bfseries\large\textcolor{white}{Next Steps for this Project}
    \end{textblock}
    % main body bullet points
    \begin{itemize}
        \item See if I can get the training to use my personal GPU/TPU
    \end{itemize}
\end{frame}
\note[itemize]{
    \item Most of this work is relegated to my ``free'' time.
    \item Have to spend my days helping people with the cloud.
}

\usebackgroundtemplate{\includegraphics[width=\paperwidth]{../images/tree.jpg}}
\begin{frame}{}
    \setlength{\TPHorizModule}{\textwidth}
    \setlength{\TPVertModule}{\textwidth}
    % Slide title in upper left
    \begin{textblock}{0.74}(0.05,0.05)
        \bfseries\large\textcolor{white}{If Time, Staff, and Money were no Object}
    \end{textblock}
    % main body bullet points
    \begin{itemize}
        \item Dreams and stretch goals.
    \end{itemize}
\end{frame}
\note[itemize]{
    \item this is the ``dreams and stretch goals'' slide.
    \item Most of this work is relegated to my ``free'' time.
    \item Have to spend my days helping people with the cloud.
}

\usebackgroundtemplate{\includegraphics[width=\paperwidth]{../images/field.jpg}}
\section{Sources and Citations}

\usebackgroundtemplate{\includegraphics[width=\paperwidth]{../images/tree.jpg}}
\begin{frame}{}
    \setlength{\TPHorizModule}{\textwidth}
    \setlength{\TPVertModule}{\textwidth}
    % Slide title in upper left
    \begin{textblock}{0.74}(0.05,0.05)
        \bfseries\large\textcolor{white}{Sources and Citations}
    \end{textblock}
    % main body bullet points
    \begin{itemize}
        \item This is an example of a list.
        \item Important business information.
    \end{itemize}
\end{frame}
\note[itemize]{
    \item point 1
    \item point 2
}

\end{document}
